% !TEX root =  ipdps18.tex

\section{Model}
\label{sec:model}
% Primary: Yves

\paragraph*{Computational Platform Model}
In this work, we consider a shared platform that comprises a set of computational
nodes, storage resources in the form of a parallel file system (PFS), and a network
that interconnects the nodes as well as the storage resources. Applications are
scheduled on the platform by a job scheduler such that computational nodes are
space-shared (dedicated) amongst concurrent application instances. However, the I/O
subsystem is time-shared (contended) amongst the application instances, \ie multiple
applications performing I/O simultaneously can result in a per-application reduction
in commit speed. Without loss of generality, we consider a straightforward linear
interference model in which the global throughput remains constant and is evenly
shared among contending applications, proportionally to their size.
(A more adversarial interference models could be substituted.)

\paragraph*{Application Workload Model}
Applications can vary in size (number of computational nodes), duration, memory
footprint and I/O requirements.  \emph{Application I/O} entails loading an input file
at startup, performing regular I/O operations during their main execution phase and
storing an output file at completion. Because applications are long-running
(typically, several hours or days) and the platform is failure-prone, applications
are protected using coordinated CR that incurs periodic \emph{CR I/O}.

To model these behavioral variations with minimal parameters, we make the following
simplifying assumptions:
\begin{compactitem}
\item There is a large number of applications, but only a small number of application
  classes, \ie, sets of applications with similar sizes, durations, footprints and
  I/O needs;
\item Other than initialization and finalization I/O, an application's regular
  (non-CR) I/O operations are evenly distributed over its makespan.
\item Job makespans are precisely known a priori. This allows us to ignore all other
  sources of job disturbance except C/R overheads.
\end{compactitem}
We used specific numbers and characteristics of application classes
based on real benchmark data, such as those provided in the APEX
workflows report on the Cielo platform~\cite{apex2016}.  To avoid the
side effects induced by hundreds of completely identical jobs,
we use normal distributions for job durations with mean
equal to original APEX value and small (20\%) standard deviation. In
the rest of the paper, we will use the term \emph{job} to denote a
specific application instance, and \emph{application class} to denote
a set of applications with similare characteristics.

\paragraph*{Checkpoint Period and I/O Interference}

Both application computation and CR generate I/O requests, and both classes of I/O
activity are scheduled using an I/O scheduling algorithm (Section~\ref{sec:algorithms}). As
described above, steady-state application I/O is regular. However, CR I/O
periodicity, $P$, depends
upon the CR policy being used.  In our model, applications either checkpoint using an
application-defined periodicity or using Young and
Daly's~\cite{young74,daly04} optimal checkpoint period. The latter interval is
computed by, $T=\sqrt{2 C \mu}$, where $C$ is the duration of the checkpoint
transfer, and $\mu$ is the application mean time between failures (MTBF).
$\mu = \frac{\muind}{q}$, where $q$ is the number of processors enrolled by the
application and $\muind$ is the MTBF of an individual
processor~\cite{springer-monograph}.  The parameters in this formula are dependent
upon application features (checkpoint dataset size) and platform features (system
reliability and I/O bandwidth).

Traditionally, when a job $J_{i}$ of class $\app{i}$ completes a checkpoint, its next
checkpoint is scheduled to happen in at least $\period{i}-\ckpt{i}$ (and the first
checkpoint is set at date $\period{i}$). With potential CR I/O interference,
the checkpoint commit may last longer than $\ckpt{i}$, and setting
the appropriate checkpointing period can be challenging.
Additionally, I/O scheduling algorithms that try to mitigate I/O interference can
impose further CR I/O delays.  In other words, the traditional strategy of scheduling
subsequent checkpoints at $\period{i}-\ckpt{i}$ yields the desired checkpointing
period $\period{i}$ only in interference-free scenarios. CR I/O delays (induced by
interferences or scheduling delays) dilate the checkpoint duration to $C_{dilated}$,
and the effective period differs from the desired period by the difference
$C_{dilated}-\ckpt{i}$.  (In Section~\ref{sec:algorithms}, we discuss how each I/O
scheduling algorithm handles this discrepancy).

\paragraph*{Job Scheduling Model}
To evaluate the scheduling policies, we consider a finite segment, typically lasting
a few days, of a representative schedule where the computing resource
usage by each application instance (job) in each class remains approximately constant at every instant. Of
course, with different job execution times, we cannot enforce a fixed proportion of
each application class at every instant. However, we ensure the proper proportion is
enforced in average throughout the schedule execution. Similarly, we enforce that at
every instant during the finite segment, at least 98\% of the nodes are enrolled for
the execution. This allows us to compare actual (simulated) performance with the
theoretical performance of a co-scheduling policy that optimizes the steady-state I/O
behavior of the job portfolio, assuming that all processors are used. We shuffle and
simultaneously present all jobs to the scheduler, which uses a simple, greedy
first-fit algorithm.  We resubmit failed jobs with a new wall-time equal to the
fraction that remained when the last checkpoint commit started. Input I/O becomes
recovery I/O; output I/O is unmodified.

\paragraph*{The Formal Model}
We consider a set $\appset$ of $\nbapps$ applications classes
$\app{1}, \ldots \app{\nbapps}$ that execute concurrently on a platform with
$\nbnodesplat$ nodes. Application class $\app{i}$ specifies:
\begin{compactitem}
\item $\nbapp{i}$: the number of jobs in $\app{i}$,
\item $\nbnodes{i}$: the number of nodes used by each job in $\app{i}$,
\item $\period{i}$: the checkpoint period of each job in $\app{i}$, and
\item $\ckpt{i}$ and $\reco{i}$: the checkpoint and recovery durations for each job in $\app{i}$ when there is no interference with other I/O operations.
\end{compactitem}
To simplify the reading, where it is not ambiguous, we extend the
notations to jobs: for a job $J_j$ belonging to application class
$\app{i}$, $\nbnodes{j} = \nbnodes{i}$, $\period{j} = \period{i}$,
$\ckpt{j} = \ckpt{i}$, and $\reco{j} = \reco{i}$ denote the number of
nodes used by job $j$, the checkpoint period of job $j$ and the
checkpoint and recovery durations of job $j$ respectfully.  At every
instant, we schedule as many jobs as possible.  Jobs that are subject
to failures are restarted at the head of the scheduling queue, so that
(given that in most cases only one node has failed and can be replaced
by a hot spare) it may restart immediately on essentially the same
compute nodes it previously occupied.
