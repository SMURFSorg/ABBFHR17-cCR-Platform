% !TEX root =  ipdps18.tex

\section{I/O Scheduling Algorithms}
\label{sec:algorithms}
% Aurelien & George

In this section, we present the application I/O scheduling algorithms used in
this study.  The first algorithm, \nocoop, represents the status-quo in which
I/O activities are scheduled independently and may incur slowdowns due to I/O
contention. The second algorithm, \fifoblock, coordinates I/O activity to
eliminate interference: I/O operations are scheduled in a
First-Come-First-Serve (FCFS) fashion and only one I/O operation executes at
any given time, while other I/O requests are blocked until their turn comes.
The third algorithm, \fifononblock, is similar except that jobs that are
waiting for the I/O token to checkpoint continue working until their turn
comes.  Lastly, we propose our heuristic, \leastwaste,  which improves on
\fifononblock by giving the I/O token to the I/O operation that will minimize
system waste. Note that unlike the blocking approaches (\nocoop and
\fifoblock), non-blocking optimizations (\fifononblock and \cooperative) may
require application code refactoring.

 % \dca{what about the
 %  case when an application must communicate before its computation can proceed?
 %  E.g. it continued execution but then arrived at a point at which it needs external
 %  input or coordination? Which brings up another question: does our model implicitly
 %  (or explicitly) handle synchronization/barrier type communication
 %  with no data?}
 % TH -> DCA: We consider a workflow batch scheduling typical of HPC
 % platforms. Applications do not communicate or synchronize with each
 % other, except through the filesystem, through their initial input
 % and final ouput. Initial input and final output are always blocking.

%Instead of following a FCFS order to select the next I/O application,
%for each requesting job, the heuristic computes the prospective
%waste incurred by delaying its I/O (considering checkpoint and
%probabilistic recovery costs, idle time, etc.) when selecting another
%job, and selects the one that minimizes the waste increase
%at the current instant.

\subsection{\nocoop I/O Scheduling}

In \nocoop I/O scheduling, jobs are executed to fill-up the system based on
processor availability, and their I/O workload (including CR activities) are
not coordinated by the system.  Instead, jobs use the parallel file system
assuming they are the sole user -- with no modifications made to their access
patterns to accommodate for possible interference. Researchers have observed
that concurrent I/O resource access can decrease the I/O bandwidth
observed~\cite{Dorier2015}.  Under the conditions of an under-provisioned I/O
substrate, our model gives each I/O stream a decrease in bandwidth linearly
proportional to the number of competing operations.  We account for the
additional delays imposed by this decreased available bandwidth as
\emph{waste}.  Since subsequent checkpoints are scheduled to start after
$\period{i}-\ckpt{i}$, and delays may result in checkpoint commit times longer
than $\ckpt{i}$, the resultant checkpoint period may be longer than
$\period{i}$. This is consistent with a trivial I/O policy that does not
consider potential contention.

% GB: The variants are described in the Simulation
% In the \nocoop algorithm, we consider two variants where checkpoints are
% tentatively taken at 1) fixed frequency (\propfixed), or 2) at the Daly frequency
% (\propdaly).

\subsection{Blocking \fifoblock FCFS I/O Scheduling}
\label{sec:fcfsblock}

A simple optimization to the \nocoop scheme is to favor one jobs' I/O over all
others. While the overall throughput may remain unchanged (given an efficient
filesystem implementation), the favored job completes its I/O workload faster
(\ie, in time $\ckpt{i}$ for a job of class $\app{i}$).  In the \fifoblock
scheme, I/O requests are performed sequentially, in request arrival order. Jobs
with outstanding I/O requests are blocked until their requests are completed.

Assuming a favorable linear interference model, a simple workload with two jobs
can show the potential advantage of the \fifoblock over \nocoop strategy.  If
the two jobs simultaneously request I/O transfers of similar data volume, $V$,
in the \nocoop strategy, both jobs take $\frac{V}{\frac{\bandavail}{2}}$ time
to complete their I/O.  In the \fifoblock strategy, the first scheduled job
takes only $\frac{V}{\bandavail}$, while the second job waits
$\frac{V}{\bandavail}$ before its own I/O starts, but then executes at full
available bandwidth completing in $\frac{2V}{\bandavail}$.  Reducing I/O
interference reduces the average I/O completion time (although fairness may be
decreased).  Once again, however, observed checkpoint durations may increase
past $\ckpt{i}$, due to I/O scheduling wait time, and the checkpointing period
may be, on average, larger than the desired $\period{i}$.

% GB: The variants are described in the Simulation
% In the \fifoblock algorithm, we again consider two variants, where checkpoints are
% tentatively taken at 1) fixed frequency (\bfifofixed), or 2) at the Daly frequency
% (\bfifodaly).

\subsection{Non-Blocking \fifononblock FCFS I/O Scheduling}
\label{sec:fcfsnonblock}

The previous strategy trades the cost of I/O interferences for idle time, as
jobs perform a blocking (idle) wait for the I/O token.  If the application
developer can refactor the program code to continue computing while awaiting
I/O request completions, it becomes possible to replace otherwise idle wait
time with useful computation. In the \fifononblock algorithm, when the previous
checkpoint ends at time $t_{now}$, a tentative time for the next checkpoint is
set at $t_{req}=t_{now}+\period{i}-\ckpt{i}$.  At time $t_{req}$, a
non-blocking I/O request is made to request the I/O token -- the I/O token is
still scheduled FCFS according to request arrival time.  The job continues its
computation until the scheduler informs it that the I/O token is available. At
this point, the job must generate its checkpoint data as soon as possible (or
after a short synchronization\footnote{In user-level checkpointing, the job
typically finishes its current computing block before generating its checkpoint
data.}).  In most applications, the granularity of the work is small enough for
a simple approach to be efficient: applications can use existing APIs in
SCR~\cite{Moody10SCR} or FTI~\cite{Bautista-Gomez11_FTI} to regularly poll if a
checkpoint should be taken at this time. In this work, we assume that this
re-synchronization cost is negligible relative to the checkpoint commit
duration.
%
%\dca{I commented out the example libraries because the described app-to-library
%  probes are different from the necessary library-to-app``callbacks'' or ``upcalls''
% needed in this case}
% See how the text was modified above.
%
Postponing checkpoint I/O increases a job's exposure to failures.  However,
if the job successfully commits the postponed checkpoint, upon a subsequent failure,
the job would restart from the time at which the postponed checkpoint was taken, not
at $t_{req}$ -- a fact that may mitigate the increased risk exposure when
compared to \fifoblock and \nocoop algorithms.

% Then, the job initiates its I/O (checkpointing, initial input, final
% output or recovery). When the active job completes its I/O, the next
% requesting job (in FCFS order) become the active I/O job.

%Aurelien: talked with Thomas and this is not what we want to study here.
% However, that
% state can be initially captured by copy-on-write mechanisms, or stored
% in local memory or in compute node-local burst buffers (\eg local SSD
% drives). Although node-local burst buffers do not offer protection
% against faults, they permit offsetting the transfer of the checkpoint
% data to a later date when the I/O token is available to the job.
% When the job finaly gets the token, the previously scratch-space
% stored checkpoint is transfered to the PFS without interference.
%NOTTODO: something about replacing with last ckpt if token doesn't come in fast enough;
% there's something that doesn't work with the T-C after C depiction: we would rollback unbounded amounts now.
% that's because we do not consider whats commented down here with local scratchpads
% the checkpoint is taken at a date t_c posterior to t_req, and we will restart at t_c, not t_req.

\subsection{Variants}
\label{sec:variants}

The periods $\period{i}$ of the checkpointing requests are input parameters to
the three strategies \nocoop, \fifoblock and \fifononblock. In
\Cref{sec:simulator}, we instantiate each strategy with two variants. The first
variant uses a fixed checkpointing period for each job, while the second
variant uses the Daly period of each job.
 
\subsection{\leastwaste Algorithm}
\label{sec:least-waste}

Finally, our \leastwaste algorithm further refines the \fifononblock algorithm
by issuing the I/O token to the job whose I/O request minimizes the total
expected waste (explained hereafter), rather than simply based on request
arrival order.  Given the time-dependent nature of this decision, the selection
may not be a global optimum, but only an approximation given currently
available information about the system status. The \leastwaste algorithm
assumes that jobs issue checkpointing requests according to their Daly
period\footnote{Fixed checkpointing makes little sense in the \leastwaste strategy,
it is designed to optimize checkpoint frequencies across all jobs.}.  For each
I/O scheduling decision, at time $t$ (when a previous I/O operation completes),
we consider a pool of $r+s$ candidates from two different categories:

\begin{compactitem}
\item Category \IOcat $\Catiocat$: Jobs $J_{i}$, $1\leq i \leq r$ with an
  (input, output or recovery) I/O request of length $v_{i}$ seconds and enrolls $q_{i}$
  processors. $J_{i}$ initiated its I/O request $d_{i}$ seconds ago and has been idle
  for $d_{i}$ seconds.

\item Category \Ckptcat $\Catckptcat$: Jobs $J_{i}$, $r+1\leq i \leq r+s$,
  with a checkpoint duration of $C_{i}$ seconds and enrolls $q_{i}$ processors.
  $J_{i}$ took its last checkpoint $d_{i}$ seconds ago and keeps executing until the
  I/O token is available for a new checkpoint. Since $J_{i}$ is a candidate,
  $d_{i} \geq \period{Daly}(J_{i})$
\end{compactitem}

If we select job $J_{i}$ to perform I/O, the expected waste $\wap{i}$
incurred to the other $r+s-1$ candidate jobs in  $\Catiocat \cup
\Catckptcat$ is computed as follows. Assume first that $J_{i} \in \Catiocat$.
Then  $J_{i}$ will use the I/O resource for $v_{i}$ seconds.
\begin{compactitem}
%
  \item Every other job $J_{j} \in \Catiocat$ will stay idle for $v_{i}$
  additional seconds, hence its waste $\wapp{i}{j}$ is $$\wapp{i}{j} = q_{j}
  (d_{j} + v_{i})$$ since there are $q_{j}$ processors enrolled in $J_{j}$ that
  remain idle for $d_{j} + v_{i}$ seconds. Note that for $J_{j} \in \Catiocat$, the
  waste $\wapp{i}{j}$ is deterministic.
%
  \item Every job $J_{j} \in \Catckptcat$ will continue executing for
  $v_{i}$ additional seconds, hence will be exposed to the risk of a failure
  that will strike within $v_{i}/2$ seconds on average. The probability of such
  a failure is $v_{i}/\mu_{j}$, where $\mu_{j} =
  \muind/q_{j}$. With this
  probability, the $q_{j}$ processors will have to recover and re-execute $d_{j} +
  v_{i}/2$ seconds of work, hence the waste $\wapp{i}{j}$ is $$\wapp{i}{j} =
  \frac{v_{i}}{\mu_{j} } q_{j} (\reco{j} + d_{j} + \frac{v_{i}}{2}) =
  \frac{v_{i}}{\muind} q^{2}_{j} (\reco{j} + d_{j} + \frac{v_{i}}{2})$$ where
  $\reco{j}$ is the recovery time for $J_{j}$. Note that for $J_{j} \in
  \Catckptcat$, the waste $\wapp{i}{j}$ is probabilistic.
%
 \end{compactitem}
 Altogether, the expected waste $\wap{i}$ incurred
to the other $r+s-1$ candidate jobs is
$$\wap{i} = \sum_{J_{j} \in \Catiocat, j\neq i} \wapp{i}{j} + \sum_{J_{j} \in \Catckptcat} \wapp{i}{j}$$
We obtain
\begin{equation}
\label{eq.selection}
\begin{array}{ll}
 \wap{i} = & v_{i} \times \left( \sum_{1 \leq j \leq r, j\neq i} q_{j} (d_{j} + v_{i}) \right.\\
& + \left. \sum_{r+1 \leq j \leq r+s}   \frac{q^{2}_{j}}{\muind} (\reco{j} + d_{j} + \frac{v_{i}}{2}) \right)
 \end{array}
\end{equation}

Assume now that the selected job $J_{i} \in \Catckptcat$. Then $J_{i}$
will use the I/O resource for $\ckpt{i}$ seconds instead of $v_{i}$
seconds for $J_{i} \in \Catiocat$. We directly obtain the counterpart
of Equation~\eqref{eq.selection} for its waste $\wap{i}$:
 \begin{equation}
\label{eq.selection2}
 \begin{array}{ll}
 \wap{i} = & \ckpt{i} \times \left( \sum_{1 \leq j \leq r} q_{j} (d_{j} + \ckpt{i}) \right.\\
& + \left. \sum_{r+1 \leq j \leq r+s, j\neq i}   \frac{q^{2}_{j}}{\muind} (\reco{j} + d_{j} + \frac{C_{i}}{2}) \right)
 \end{array}
\end{equation}

Finally, we select the job $J_{i} \in \Catiocat \cup \Catckptcat$
whose waste $\wap{i}$ is minimal. 
