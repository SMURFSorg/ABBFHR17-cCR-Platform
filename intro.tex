% !TEX root =  ipdps18.tex

\section{Introduction}
\label{sec:intro}
% Primary: George & Dorian

%space sharing but not quite
\emph{Space-sharing} high-performance computing (HPC) platforms for the
concurrent execution of multiple parallel applications is the prevalent usage
pattern in today's HPC centers.  In fact, space-sharing in this fashion is more
common than \emph{capability} workloads that span the entire
platform~\cite{Weidner2016}. Furthermore, while computational nodes are
dedicated to a particular application instance, the interconnect links and
storage partition are typically shared amongst application instances. Therefore,
without careful consideration, network and storage contention can reduce
individual application and overall system performance \todo{kbf: CITE "There goes the
neighborhood"?}.

On these platforms, checkpoint/restart (CR) is the most common strategy
employed to protect applications from underlying faults and failures.
Generally, CR periodically outputs snapshots (i.e. checkpoints) of its global,
distributed state to some stable storage device. When an application failure
occurs, the last stored checkpoint is retrieved and used to restart the
application.  Typically, concurrently executing applications independently
decide when to take their own checkpoints.

There are two widely-used approaches to determine when an application should
\emph{commit} a checkpoint: (i)~using a fixed checkpoint period (typically one
or a few hours) for each application; and (ii)~using platform and
application-specific metrics to determine its optimal checkpoint period. In the
second approach, the well-known Young/Daly formula~\cite{young74,daly04} yields
an application optimal checkpoint period, $\sqrt{2 \mu C}$ seconds, where $C$
is the time to commit a checkpoint and $\mu$ the application Mean Time Between
Failures (MTBF) of the platform.  In most cases, $\mu = \frac{\muind}{q}$,
where $q$ is the number of processors enrolled by the application and $\muind$
is the MTBF of an individual processor~\cite{springer-monograph}. Therefore,
both $\mu$ and $C$ in the Young/Daly formula are application-dependent, and
optimal periods can be quite different over the application spectrum.

Independent CR of concurrent application instances can incur significant
resource wastage, because they lead to an inefficient usage of an already
scarce resource, namely available I/O bandwidth\todo{kbf: CITE R. Ross SOP
paper?}.  There are two major reasons for this:

\begin{enumerate}
        
\item \emph{Application-CR I/O contention}: On many systems, the I/O subsystem
does not have enough available usable bandwidth to meet the requirements of
the concurrent application workloads\todo{kbf: CITE R. Ross SOP paper?}. This
congestion is expected to worsen going forward with the emergence of many data
intensive workloads in HPC\todo{kbf: CITE}.  Let $\bandtotal$ be the total
filesystem I/O bandwidth.  Concurrently executing applications typically
perform regular (non-CR) I/O operations throughout their execution, so that
only a fraction $\bandavail$ of the total bandwidth remains available for
checkpoints.  This fraction may be insufficient, particularly when some
applications perform intensive non-checkpoint I/O and others may write very
large checkpoints.
  % $\bandtotal$ of a well-provisioned platform should allow for efficient CR
  % I/O activities.

\item \emph{CR-CR I/O contention}: Most importantly, there is a high
probability of overlapping CR activity amongst concurrent application
instances.  Consider the simple case where two applications of same size
checkpoint simultaneously a file of the same size. Each will be assigned half
the fraction $\bandavail$ to checkpoint, therefore the commits will take twice
as long. Such interferences can severely decrease application efficiency and
overall platform throughput (e.g., when the expected checkpoint commit time
used to compute the optimal checkpoint interval differs from the actual
checkpoint commit time)\todo{footnote?}.

\end{enumerate}

In this work, we develop and investigate a cooperative CR scheduling strategy
for concurrently executing HPC applications.  Our objective is to assess the
impact of such interferences and to design scheduling algorithms that optimize
I/O bandwidth availability for CR activity.  Using these cooperative
algorithms, applications checkpoint sequentially, with a dynamic,
priority-dependent frequency dictated by a cooperative scheduler.  When enough
I/O bandwidth is available, each application checkpoints with its optimal,
Young/Daly, period.  However, when I/O bandwidth is scarce, our scheduling
algorithm provides an optimal checkpoint period that maximizes overall platform
throughput. This cooperative checkpoint process is calculated such that there
is no I/O interference and minimal re-work to be done when failures occur.

Therefore, the main contributions of this paper are the following:

\begin{itemize}

\item Development of a model allowing for the quantification of
the I/O interference of checkpointing applications sharing a common underlying I/O
substrate.

\item Investigation of the costs of various I/O-aware scheduling
strategies through both steady-state analysis as well as detailed simulations.

\item A detailed survey of a number scheduling strategies: from oblivious
algorithms similar to  those currently deployed on many large-scale platforms,
to ones which exploit application knowledge in an effort to  minimize the total
system waste by scheduling the application with the most critical I/O needs.

\end{itemize}

% - a model to predict the shared I/O impact on multiple applications scenarios
% - I/O scheduling algorithms for non-cooperative application scheduling
%   - non-cooperative I/O scheduling: apps are selected to fill the gaps based on processor count (traditional approach)
%   - blocking FIFO I/O scheduling: favor one of the I/O application
%   - non-blocking FIFO I/O scheduling: same as above but the cost of the queueing the app is now independent of the interference pattern
%   - least-waste algorithm: select the app that will minimize the system waste (I/O or C/R candidate)
% - steady state analysis
% - simulation
% - results

The rest of the paper is organized as follows. Our model is described in
\Cref{sec:model}, followed by a description of the various scheduling
strategies in \Cref{sec:algorithms}. \Cref{sec:lowerbound} presents a
theoretical analysis of the model under a steady-state scenario, and provides a
lower bound of the optimal platform waste. \Cref{sec:simulator} describes the
discrete event simulator used to quantitatively compare the different
scheduling strategies.  \Cref{sec:results} presents the results of the
simulation, providing guidance on the necessary I/O bandwidth for  current and
future systems. This work concludes with related works described in \Cref{sec:related},
followed by a summary and future directions outlined in \Cref{sec:conclusion}.

% - Section~\ref{sec:model} describe the scenario under investigation
% - Section~\ref{sec:algorithms} describe the different I/O scheduling
%   algorithms that we plan to analyze, including one that is highly related to
%   the default scheduling on most HPC platforms
% - Section~\ref{sec:lowerbound} describe a theoretical scenario that allow us
%   to derive the lower-bound
% - Section~\ref{sec:simulator} describe the simulator used to validate the
%   results
% - Section~\ref{sec:results} present the results
% - Section~\ref{sec:related} depicts the related work field
% - Section~\ref{sec:conclusion} conclude
