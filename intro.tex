% !TEX root =  ipdps18.tex

\section{Introduction}
\label{sec:intro}
% Primary: George & Dorian

%space sharing but not quite
\emph{Space-sharing} high-performance computing (HPC) platforms for the concurrent
execution of parallel applications is today's de-facto standard HPC execution
mode. In fact, space-sharing in this fashion is more common than \emph{capability
  application runs} that span entire platforms~\cite{xx}. However, while
computational nodes are completely dedicated to particular application instances,
most often the networks that interconnect computational nodes, as well as storage
devices, are shared amongst application instances. In other words, without careful
consideration, network and storage contention can reduce individual application and
overall system performance.

In HPC, checkpoint/restart (CR) is the most common strategy employed to protect
applications from underlying faults and failures. Generally, a parallel application
protected by CR periodically outputs snapshots (a.k.a checkpoints) of its global,
distributed state to some remote storage device. If an application failure occurs, a
stored checkpoint is retrieved and used to restart the application from the
intermediate point captured by the checkpoint. Typically, concurrently executing
applications independently (non-cooperatively) decide when to take their own
checkpoints.

There are two widely-used, non-cooperative approaches to determine when an
application should \emph{commit} a checkpoint: (i) using a fixed checkpoint period
(typically one hour) for each application; and (ii) using application-specific
features to determine its checkpoint period. In the second approach, the well-known
Young/Daly formula~\cite{young74,daly04} yields an application optimal checkpoint
period, $\sqrt{2 \mu C}$ seconds, where $C$ is the time to commit a checkpoint and
$\mu$ the application Mean Time Between Failures (MTBF). We have $\mu = \frac{\muind}{q}$,
where $q$ is the number of processors enrolled by the application and $\muind$ is the
MTBF of an individual processor~\cite{springer-monograph}. As a consequence, both
$\mu$ and $C$ in the Young/Daly formula are application-dependent, and optimal
periods can be quite different over the application spectrum.

Non-cooperative CR of concurrent application instances can incur significant resource
wastage, because they lead to an inefficient usage of an already scarce resource,
namely the I/O system.  There are two major reasons for this:
\begin{enumerate}
\item \emph{Application-CR I/O contention}: As a basis, the I/O subsystem already may
  not be capable of sustaining the I/O requirements of all the concurrent
  applications.  Let $\bandtotal$ be the total filesystem I/O bandwidth.  The
  concurrently executing applications perform regular (non-CR) I/O operations
  throughout their execution, so that only a fraction $\bandavail$ of the total
  bandwidth remains available for checkpoints.  This fraction may be insufficient,
  particularly when some applications perform intensive non-checkpoint I/O and others
  have large checkpoint.
  % $\bandtotal$ of a well-provisioned platform should allow for efficient CR I/O
  % activities.

\item \emph{CR-CR I/O contention}: More importantly, there is a high probability of
  coincident CR activity amongst concurrent, non-cooperative application instances.
  Consider the simple case where two applications checkpoint simultaneously a file of
  the same size. Each will be assigned half the fraction $\bandavail$ to checkpoint,
  so that both checkpoints will be twice longer than expected. Such interferences
  severely decrease application efficiency (e.g., when the expected checkpoint commit
  time used to compute the optimal checkpoint interval differs from the actual
  checkpoint commit time) and overall platform throughput.
\end{enumerate}

In this work, we study cooperative CR scheduling strategies for concurrently
executing HPC applications.  Our objective is to assess the impact of such
interferences and to design scheduling algorithms that optimize I/O bandwidth
availability for CR activity.  Using our cooperative algorithms, applications checkpoint
sequentially, with a dynamic, priority-dependent frequency dictated by a cooperative
scheduler.  When enough I/O bandwidth is available, each application checkpoints with
its optimal, Young/Daly, period. However, when I/O bandwidth is scarce, our
scheduling algorithm provides an optimal checkpoint process that maximizes platform
throughput. This cooperative checkpoint process is calculated such that there is no
I/O interference and minimal re-work to be done when failures occur.

The main contributions of the paper are to provide a model allowing to quantify
the I/O interferences for applications sharing a common underlying I/O
subsystem, and to estimate the costs on potential scientific throughput of different I/O-aware
scheduling strategies, through
a steady-state analysis as well as detailed simulations. 
The scheduling strategies included in this study cover a
large range of I/O behaviors, going from non-cooperative algorithms, as those
currently deployed on most large-scale platforms, to ones that take advantage of
application knowledge to minimize the total system waste by scheduling the
application with the most critical I/O needs.
% - a model to predict the shared I/O impact on multiple applications scenarios
% - I/O scheduling algorithms for non-cooperative application scheduling
%   - non-cooperative I/O scheduling: apps are selected to fill the gaps based on processor count (traditional approach)
%   - blocking FIFO I/O scheduling: favor one of the I/O application
%   - non-blocking FIFO I/O scheduling: same as above but the cost of the queueing the app is now independent of the interference pattern
%   - least-waste algorithm: select the app that will minimize the system waste (I/O or C/R candidate)
% - steady state analysis
% - simulation
% - results

The rest of the paper is organized as follows. The model assumed by our study is
described in  Section~\ref{sec:model}, followed, in
Section~\ref{sec:algorithms}, by an exhaustive description of the different
scheduling strategies under consideration. The Section~\ref{sec:lowerbound}
provide a theoretical analysis of the model under a steady-state scenario, and
provides a necessary condition to reach a lower bound of the waste. The
Section~\ref{sec:simulator} presents the discrete event simulator used to
quantitatively compare the different scheduling strategies using a well-defined
set of applications from LANL. Section~\ref{sec:results} presents the results of
the simulation, on current systems, and provide hints of the necessary I/O
bandwidth on prospective systems. The paper completes by describing a set of
related works in Section~\ref{sec:related}, followed in
Section~\ref{sec:conclusion} by a conclusion and directions for future work.

% - Section~\ref{sec:model} describe the scenario under investigation
% - Section~\ref{sec:algorithms} describe the different I/O scheduling algorithms that we plan to analyze, including one that is highly related to the default scheduling on most HPC platforms
% - Section~\ref{sec:lowerbound} describe a theoretical scenario that allow us to derive the lower-bound
% - Section~\ref{sec:simulator} describe the simulator used to validate the results
% - Section~\ref{sec:results} present the results
% - Section~\ref{sec:related} depicts the related work field
% - Section~\ref{sec:conclusion} conclude
