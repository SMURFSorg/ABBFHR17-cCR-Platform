
% !TEX root =  ipdps18.tex

\section{Related Work}\label{sec:related}
% Primary: Kurt

We first discuss papers related to I/O
pressure due to checkpointing, followed by those related to avoiding I/O
interference.  We note that these techniques are not necessarily independent;
reducing I/O pressure will generally reduce the likelihood of interference.
Therefore, we attempt to limit our I/O interference discussion to those
techniques which consider the global scheduling of checkpoints and/or general
I/O across a platform.

%\todo[inline]{kbf: I am unsure about this breakdown.  These two things do not
%seem independent; reducing pressure seems to al reduce interference ...}

\paragraph*{Checkpointing and I/O}

For a single application, the optimal checkpointing period is given by the
Young/Daly formula~\cite{young74,daly04}. This period minimizes the platform
waste, defined as the fraction of the execution time that does not contribute
to the progress of the application (the time \emph{wasted}).  There are two
sources of waste, the time spent taking checkpoints (which calls for longer
checkpoint periods), and the time needed to recover and re-execute after each
failure (which calls for shorter checkpoint periods), The Young/Daly period
achieves the optimal trade-off between both sources to minimize the total
waste. Arunagiri et al.~\cite{Arunagiri2010} studied the use of a longer,
sub-optimal period with the intent of reducing I/O pressure, and shown,
both analytically and using four real-life platforms, that a decrease in the
I/O requirement can be achieved with only a small increase in waste.

\paragraph*{Reducing I/O Pressure}

Optimizations that target reducing I/O pressure from a single application can
generally be divided into two classes; those that attempt to hide or reduce the
checkpoint commit times without reducing the volume of data, and those that
reduce commit times by reducing checkpoint volumes.

Strategies that attempt to hide checkpoint times include
Diskless~\cite{Plank98Diskless} and remote checkpoint
protocols~\cite{Cornwell11RemoteBLCR}
which leverage the typically higher available bandwidths of the network or
other storage media like RAM in order to mitigate the performance of slower
storage media like spinning or solid-state disks. Additionally, remotely stored
checkpoints have the additional benefit of allowing systems to survive
non-transient node failures. Similarly, multi-level checkpoint protocols like
SCR~\cite{Moody10SCR,Vaidya95TwoLevel} attempt to hide checkpoint commit times
by writing checkpoints to RAM, flash storage, or local disk on the compute
nodes~\cite{Kougkas2017} in addition to the parallel file system thereby
improving checkpoint or general I/O bandwidth.  Finally, checkpoint-specific
file systems like PLFS~\cite{Bent09PLFS} leverage the I/O patterns and
characteristics specific to checkpoint data to optimize checkpoint data
transfers to/from parallel file systems and therefore reduce checkpoint commit
times.

The strategies which attempt to reduce checkpoint sizes include \emph{memory
exclusion} which leverage user-directives or other hints to exclude portions of
process address spaces from checkpoints~\cite{Plank99MemoryExclusion}.
Additionally, incremental checkpointing protocols reduce checkpoint volumes by
utilizing the OS's memory page protection facilities to detect and save only
pages that have been updated between consecutive
checkpoints~\cite{Bronevetsky09Compiler,
Chen97CLIP,Elnozahy92ConsistentCheckpointing,Li94ConcurrentCheckpointing,
Plank94Libckpt,Paun10IncrementalWeibull,Kiswany08stdchk}.  Similarly,
page-based hashing techniques can also be used to avoid checkpointing pages
that have been written to but whose content has not
changed~\cite{Ferreira11Libhashckpt}.  Finally, compression-based techniques
use standard compression algorithms to reduce checkpoint
volumes~\cite{Ibtesham12Compression} and can be used at the
compiler-level~\cite{Li90CATCH} or in-memory~\cite{Plank94ICKP}.  Related,
Plank et al. proposed \textit{differential compression} to reduce checkpoint
sizes for incremental checkpoints~\cite{Plank95CompressedDiff} and Tanzima et
al.  show that similarities amongst checkpoint data from different processes
can be exploited to compress and reduce checkpoint data
volumes~\cite{tanzima12mcrengine}.  Finally, Sasaki et al propose using a lossy
compression method based on wavelet transform and vector quantization to the
checkpoints of a production climate application~\cite{sasaki2015}, while Ni et
al~\cite{Ni2014} study the trade-offs between the loss of precision, compression
ratio, and application correctness due to lossy compression.

\paragraph*{Avoiding I/O interference}

Most closely related to our work, a number of studies have considered the
global scheduling of checkpoints and other I/O across the platform to reduce
the overall congestion therefore increase performance.  Aupy et
al.~\cite{Aupy:2017:Periodic} presented a decentralized I/O scheduling
technique for minimizing the congestion due to checkpoint interference by
taking advantage of the observed periodic and deterministic nature of HPC
application checkpoints and I/O.  This technique allows the job scheduler to
pre-define each application’s I/O behavior for their entire execution.
Similarly, a number of works have investigated the efficiency of online
schedulers for data intensive~\cite{Groot2013,Sim:2015:AnalyzeThis} and HPC
workload
I/O~\cite{Dorier2015,Gainaru:2016:Scheduling,Zhou:2015:IOAware,Herbein2017}.
Finally, a number of works have investigated utilizing system reliability
information recorded on the system~\cite{Oliner:2006:Cooperative} and the
statistical properties of these failures~\cite{Tiwari:2014:Lazy} to determine
effective checkpoint intervals for the portion of the system used by the
workload.

\paragraph*{Summary}

The present work distinguishes itself from these previous studies in a number
of important ways.  First, this technique considers existing platform I/O, a
number of previous studies do not.  Unlike some previous work, this method is
agnostic to the I/O patterns of the considered applications as long as they are
known.  Also, this technique attempts to optimize the efficiency of the entire
platform, with the changing workloads and failures running on that platform,
rather than just considering one workload. Finally and most importantly, this
approach provides optimal checkpointing periods in environments where I/O is
highly constrained and Daly/Young's formula is not appropriate, a common
scenario on many leadership-class systems.
