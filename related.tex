
% !TEX root =  ipdps18.tex

\section{Related Works}\label{sec:related}
% Primary: Kurt

We survey related work in this section. We first discuss papers related to I/O pressure
due to checkpointing. Then we deal with I/O interference.



\subsection{Checkpointing and I/O}

For a single application, the optimal checkpointing period is given by the Young/Daly
formula~\cite{young74,daly04}. This period minimizes the platform waste, defined as
the fraction of the
execution time that does not contribute to the progress of the application (the
time \emph{wasted}).  There are two sources of waste, the time spent taking checkpoints
(which calls for longer checkpoint periods),
and the time needed to recover and re-execute after each failure
(which calls for shorter checkpoint periods),
The Young/Daly
period achieves the optimal trade-off between both sources to minimize the
total waste.
However, this optimal period may put too much pressure
on the I/O system. It is possible to use a longer, sub-optimal, period that would incur
less pressure and still lead to a reasonable waste. S. Arunagiri et al.~\cite{Arunagiri2009} have studied such trade-offs and they have shown, both analytically and instantiating the model with four real-life platforms,
that a great decrease in I/O requirement can be achieved  at the price of a small increase of the waste.

\subsection{I/O interference}
