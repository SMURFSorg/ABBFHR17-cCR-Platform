\begin{abstract}
% Primary: George & Dorian
  In high-performance computing environments, input/output (I/O) from various
  sources often contend for I/O bandwidth or network resources. Without careful
  consideration, contending I/O from independently operating sources can lead to
  significant performance degradation, especially for capacity I/O loads such as
  the I/O from checkpoint/restart (CR) services used to protect these
  computations from platform faults.
%   For example, I/O from concurrently running applications can contend with each
%   other as well as with other I/O loads, such as the I/O from
%   checkpoint/restart (CR) services used to protect these computations from
%   platform faults.
  In this work, we consider a cooperative I/O scheduling policy that optimizes
  the way concurrently executing, CR-based applications share I/O resources.
  Using this cooperative policy, application checkpoints are cooperatively
  orchestrated to prevent congestion and to minimize the global waste. Our
  results show that the optimal checkpoint interval as defined by Young/Daly
  provide a sensible metric for a single applications, but fails to address a
  larger problem, the resource contention at the platform scale.  Other,
  potentially suboptimal, I/O scheduling can provide a significant improvement
  on the overall application performance, maximizing the platform throughput.
%   sequentially, with a dynamic, priority-dependent frequency dictated by the
%   scheduler. When enough I/O bandwidth is available, each application
%   checkpoints with its optimal period. However, when I/O bandwidth is scarce,
%   our scheduling algorithm provides an optimal checkpoint process that
%   maximizes platform throughput. Our results show ...
\end{abstract}
