\begin{abstract}
% Primary: George & Dorian
  In high-performance computing environments, input/output (I/O) from various
  sources often contend for I/O or network resources. Adding to these
  traditional I/O operations the I/O loads resulting from checkpoint/restart
  (CR) services used to protect the parallel applications from platform faults,
  increases the contention and will lead to an additional increase in platforms
  performance degradation.
%  aggravates the problem. Without careful consideration, contending I/O from
%  independently operating sources will lead to significant performance
%  degradation.
% especially for capacity I/O loads such as
%  the I/O from checkpoint/restart (CR) services used to protect these
%  computations from platform faults.
%   For example, I/O from concurrently running applications can contend with each
%   other as well as with other I/O loads, such as the I/O from
%   checkpoint/restart (CR) services used to protect these computations from
%   platform faults.
  In this work, we consider how a cooperative scheduling policy that optimizes
  the way concurrently executing CR-based applications share I/O resources,
  would impact congestion, and therefore performance. We provide a theoretical
  model, and derive a set of necessary constraints to minimize the global waste.
%  the scientific throughput of these platforms. Using
%  this cooperative policy, application checkpoints are cooperatively
%  orchestrated to prevent congestion and to minimize the global waste.
   Our results show that the optimal checkpoint interval as defined by
   Young/Daly provide a sensible metric for a single application, but fails to
   address a larger problem, the resource contention at the platform scale.  We
   demonstrate how other, I/O scheduling strategies can provide a significant
   improvement on the overall application performance, maximizing the platform
   throughput.
%   sequentially, with a dynamic, priority-dependent frequency dictated by the
%   scheduler. When enough I/O bandwidth is available, each application
%   checkpoints with its optimal period. However, when I/O bandwidth is scarce,
%   our scheduling algorithm provides an optimal checkpoint process that
%   maximizes platform throughput. Our results show ...
\end{abstract}
