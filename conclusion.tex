\section{Conclusion and Future Work}
\label{sec:conclusion}
% Primary: Aurélien

In this paper we presented a comprehensive model to capture interference
between multiple applications performing fault-tolerance related I/O
on a shared HPC system. We proved that ... We designed multiple algorithms
to schedule and order the checkpointing I/O workload, with the intent of
diminishing the average slowdown sustained by applications on the
platform induced by sharing the I/O subsystem, \ie improve the throughput
of the platform. We formulated a steady-state analysis that ...
We designed a event-based simulator that permits
executing typical HPC workloads on current and prospective systems.
With this simulator we have been able to demonstrate that our proposed
heuristic improves the platform efficiency. Unsurprisingly the gain is
more marked on platforms with a challenging MTBF or with
under-provisioned I/O, but our heuristic improves the efficiency in
all cases.  We also simulated the situation on a not yet available
platform, this time with the goal of providing guidance in the
general I/O requirements for future HPC systems to be able to
sustain checkpointing with the desired 80\% efficiency, a goal that
we have found achievable with ... TB/s of I/O bandwidth.

As burst-buffers and other node-local NVRAM storage are becoming
more common, a natural extension of this work is to consider the effect
 of I/O contention/interference with hierarchies, in which subgroups of nodes
may share a burst buffer at a cabinet level; and thus experience
interference for I/O in that same cabinet, but be immune to interferences
from I/O from other cabinets. Another interesting point in that context
is to consider the limit on the amount of data a burst-buffer may hold,
and how it interplays with checkpointing activities in the case where the burst
buffers may not be committed to the sink PFS rapidly enough (possibly
because of interference between multiple burst-buffers being
flushed to the sink PFS simultaneously) to sustain the application
optimal checkpointing strategy. Again, we expect that scheduling the
commits to the PFS sink with an heuristic that considers the cost of
lost computation in failure cases can play a role in improving the
efficiency of the whole system in that case.
